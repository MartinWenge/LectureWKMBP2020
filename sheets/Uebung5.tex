%% LyX 2.3.4.4 created this file.  For more info, see http://www.lyx.org/.
%% Do not edit unless you really know what you are doing.
\documentclass[12pt,german,journal=mamobx,manuscript=article,maxauthors=15,biblabel=plain]{article}
\usepackage[T1]{fontenc}
\usepackage[latin9]{inputenc}
\usepackage[a4paper]{geometry}
\geometry{verbose,tmargin=1in,bmargin=1in,lmargin=1in,rmargin=1in,headheight=0.5cm,headsep=0.5cm,footskip=0.5cm}
\setlength{\parskip}{\smallskipamount}
\setlength{\parindent}{0pt}
\usepackage{amsmath}

\makeatletter
%%%%%%%%%%%%%%%%%%%%%%%%%%%%%% User specified LaTeX commands.
\usepackage{babel}

\makeatother

\usepackage{babel}
\begin{document}

\section*{�bung 5}

\subsubsection*{Adsorption von Monomeren}

Wir betrachten ein System mit Teilchen (Monomere) ohne paarweise Wechselwirkungen.
Die Simulationsbox sei in zwei Raumrichtungen periodisch, in einer
nicht-periodisch. Diese Wand hat eine attraktive Wechselwirkung mit
den Monomeren, wobei die Wechselwirkungsst�rke durch die Energie pro
Monomer-Wand-Kontakt $\delta$ gegeben ist. Sie k�nnen das Modul lattice\_gas.py
benutzen.
\begin{enumerate}
\item Erstellen Sie eine Startkonformation mit $256$ einzelnen Monomeren
in einer kubischen Box mit $32^{3}$ Gitterpl�tzen.
\item Implementieren Sie absorbierenden W�nde um diese Monomere an zwei
gegen�berliegenden W�nden mit verschiedenen Adsorptionsenergien $\delta\in[0,10]$
zu simulieren. Nutzen Sie dazu den Metropolis Algorithmus. Messen
Sie die Anzahl adsorbierter Monomere $N_{\text{ads}}$.
\item Tragen Sie die Adsorptionsisotherme $N_{\text{ads}}/N(\delta)$ auf
und vergleichen Sie mit der analytischen Vorhersage der Boltzmann
Statistik: 
\[
\frac{N_{\text{ads}}}{N}(\delta)=\frac{\exp(\delta/k_{\text{B}}T)}{\exp(\delta/k_{\text{B}}T)+V_{0}/V_{\text{ads}}}
\]
Dabei ist $N$ die Gesamtanzahl der Teilchen in der Box mit dem Gesamtvolumen
$V$. $V_{\text{ads}}$ bezeichnet das Volumen, das f�r die Adsorption
zur Verf�gung steht, $V_{0}$ das Volumen, das nicht f�r die Adsorption
zur Verf�gung steht.
\item Leiten Sie diesen Ausdruck her. Nehmen Sie als Ausgangspunkt die Entropie
des idealen Gases $S=S(T)-k_{\text{B}}N\ln(V/v_{0}N)$, wobei $v_{0}$
das Eigenvolumen eines Monomeres ist.
\end{enumerate}

\subsection*{Adsorption von Polymerketten}

Wir betrachten ein BFM-Polymer (siehe �bung 4) in einer Simulationsbox
mit zwei periodischen und einer nicht-periodischen Raumrichtung (z-Richtung).
Die Polymerkette soll mit ausgeschlossenem Volumen simuliert werden.
Sie k�nnen das modul bfm\_simulator.py benutzen.
\begin{enumerate}
\item Erstellen Sie eine Einzelkette mit Kettenl�nge $N=32$ in einer kubischen
Box mit $64^{3}$ Gitterpl�tzen in starker Adsorption ($\delta=10\;k_{\text{B}}T$).
Zeigen Sie, dass f�r $N_{\text{ads}}/N\approx1$ die Kette einem zweidimensionalen
self avoiding walk entspricht ($R_{\text{g,z}}\approx0$).
\item Simulieren Sie eine L�sung linearer Ketten der L�nge $N=128$ in einer
Box mit den Abmessungen $[64,64,196]$ und mit einer Volumenkonzentration
von $\phi=0.025$ in moderater Adsorption ($\delta\le1$ $k_{\text{B}}T$).
Berechnen Sie das Dichteprofil zwischen den nicht-periodischen W�nden
$\phi(z)$ und vergleichen Sie mit dem Modell des de Gennes self similar
carpet. 
\item \textbf{Extra:} Zeigen Sie f�r die Einzelkette in starker Adsorption,
dass der Gyrationsradius in $xy$-Richtung wie ein 2D Self-Avoiding
Walk skaliert ($R_{\text{g 2D}}\sim N^{3/4}$)
\end{enumerate}

\end{document}
