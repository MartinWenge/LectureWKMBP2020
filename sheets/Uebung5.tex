%% LyX 2.3.4.4 created this file.  For more info, see http://www.lyx.org/.
%% Do not edit unless you really know what you are doing.
\documentclass[12pt,german,journal=mamobx,manuscript=article,maxauthors=15,biblabel=plain]{article}
\usepackage[T1]{fontenc}
\usepackage[latin9]{inputenc}
\usepackage[a4paper]{geometry}
\geometry{verbose,tmargin=1in,bmargin=1in,lmargin=1in,rmargin=1in,headheight=0.5cm,headsep=0.5cm,footskip=0.5cm}
\setlength{\parskip}{\smallskipamount}
\setlength{\parindent}{0pt}
\usepackage{amsmath}

\makeatletter
%%%%%%%%%%%%%%%%%%%%%%%%%%%%%% User specified LaTeX commands.
\usepackage{babel}

\makeatother

\usepackage{babel}
\begin{document}

\section*{�bung 5}

\subsubsection*{Adsorption von Monomeren}

Wir betrachten ein System mit nicht wechselwirkenden Teilchen (Monomere).
Die Simulationsbox sei in zwei Raumrichtungen periodisch, in einer
nicht-periodisch. diese Wand soll eine attraktive Wechselwirkung mit
den Monomeren haben, wobei die Wechselwirkungsst�rke durch die Energie
pro Monomer-Wand ($z=0,$Boxgr��e$-1$) Kontakt $\epsilon$ gegeben
ist.
\begin{enumerate}
\item Erstellen Sie eine Startkonformation mit 512 einzelnen Monomeren in
einer kubischen Box mit $64^{3}$ Gitterpl�tzen.
\item Implementieren Sie absorbierenden W�nde um diese Monomere an zwei
gegen�berliegenden W�nden mit verschiedenen Adsorptionsenergien $\epsilon[0,10]$
zu simulieren. Messen Sie die mittlere Anzahl adsorbierter Monomere.
\item Tragen Sie die Adsorptionsisotherme $N_{ads}/N(\epsilon)$ auf und
vergleichen Sie mit der analytischen Vorhersage der Boltzmann Statistik
\[
\frac{N_{ads}}{N}(\epsilon)=\frac{\exp(\epsilon/k_{\text{B}}T)}{\exp(\epsilon/k_{\text{B}}T)+V_{0}/V_{abs}}
\]
\item Leiten Sie diesen Ausdruck her. Nehmen Sie als Ausgangspunkt die Entropie
des idealen Gases $S=S(T)-k_{\text{B}}N\ln(V/v_{0}N)$. 
\end{enumerate}

\subsection*{Adsorption von Polymerketten}

Wir betrachten ein BFM-Polymer (siehe �bung 4) in einer kubischen
Simulationsbox mit $64^{3}$ Gitterpl�tzen mit zwei periodischen und
einer nicht-periodischen Raumrichtung. Die Polymerkette soll mit ausgeschlossenem
Volumen simuliert werden.
\begin{enumerate}
\item Erstellen Sie eine Einzelkette mit Kettenl�nge $N=64$ und geeigneten
Boxgr��en f�r Simulationen mit absorbierenden W�nden.
\item Bestimmen Sie die Adsorptionsenergie f�r die Kettenl�nge $N=64$ f�r
$\epsilon[0,2]$ und erl�utern Sie Ihre Beobachtungen. Vergleichen
Sie (qualitativ) mit dem idealen Gas.
\item Simulieren Sie Ketten unterschiedlicher L�nge in schwacher ($\epsilon<1$
$k_{\text{B}}T$) und starker Adsorption ($\epsilon>1k_{\text{B}}T$)
und bestimmen Sie einerseits das Verhalten der Ausdehnung $R_{g}\sim N^{x}$
parallel und senkrecht zur Wand. Vergleichen Sie mit den theoretischen
Vorhersagen.
\end{enumerate}

\end{document}
