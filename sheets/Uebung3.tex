%% LyX 2.3.4.4 created this file.  For more info, see http://www.lyx.org/.
%% Do not edit unless you really know what you are doing.
\documentclass[12pt,german,journal=mamobx,manuscript=article,maxauthors=15,biblabel=plain]{article}
\usepackage[T1]{fontenc}
\usepackage[latin9]{inputenc}
\usepackage[a4paper]{geometry}
\geometry{verbose,tmargin=1in,bmargin=1in,lmargin=1in,rmargin=1in,headheight=0.5cm,headsep=0.5cm,footskip=0.5cm}
\setlength{\parskip}{\smallskipamount}
\setlength{\parindent}{0pt}
\usepackage{amsmath}

\makeatletter
%%%%%%%%%%%%%%%%%%%%%%%%%%%%%% User specified LaTeX commands.
\usepackage{babel}

\makeatother

\usepackage{babel}
\begin{document}

\section*{�bung 3}

\subsection*{Diffusion und Zufallspfade (random walks)}

Schreiben Sie ein Programm, mit dem Sie Zufallspfade in 2D erzeugen
k�nnen. Erstellen Sie Funtionen f�r die Berechnung des End-zu-End
Abstands $R_{\text{e}}$ und des Gyrationsradius $R_{\text{g}}$.
Nehmen Sie an, dass sich der L�ufer bei jedem Schritt um eine Einheit
in genau eine der beiden Raumrichtungen bewegt.

\subsubsection*{Simple Sampling mit Zufallspfaden}
\begin{enumerate}
\item Erzeugen Sie 1000 Zufallspfade aus 512 Schritten und bestimmen Sie
die H�ufigkeitsverteilung der Endpositionen der Zufallspfade. Vergleichen
Sie die entstehende Verteilung mit den erwarteten Verteilungen in
1D, sowie radial.
\item Berechnen Sie die mittleren End-zu-End Abst�nde $R_{\text{e}}$ und
Gyrationsradien $R_{\text{g}}$ einer Serie an Zufallspfaden verschiedener
L�nge (etwa $2^{4}$-$2^{11}$), um den Zusammenhang $R^{2}=b^{2}N$
zu �berpr�fen. Welche Werte erhalten Sie jeweils f�r die Bindungsl�nge
$b$?
\end{enumerate}

\subsubsection*{Importance Sampling mit Zufallspfaden}

Erweitern Sie Ihr Programm um eine Funktion, die einen bestehenden
Zufallspfad durch Verschiebung einzelner Schritte/Knoten ver�ndert
und dabei einen neuen, zul�ssigen Zufallspfad erzeugt. Dabei darf
die L�nge des Pfades nicht ver�ndert werden, weder die Anzahl der
Schritte, noch die L�nge eines Schrittes.
\begin{enumerate}
\item Bestimmen Sie die Autokorrelationsfunktion des End-zu-End Vektors
\begin{align*}
c(\Delta t)=\left\langle \overrightarrow{R_{\text{e}}}(t_{0})\cdot\overrightarrow{R_{\text{e}}}(t_{0}+\Delta t)\right\rangle /\left\langle \vec{R_{\text{e}}^{2}}\right\rangle 
\end{align*}
einer Kette aus 64 Bindungsvektoren, die f�r eine gro�e Zahl an verschiedenen
Systemen ($\ge1000$) als Funktion des Zeitintervalls $\Delta t$
gemittelt wird. Wie gro� ist die Relaxationszeit und f�r welche $\Delta t$
erh�lt man statistisch unkorrelierte Konformationen?
\item Wiederholen Sie die Berechnungen von $R_{\text{e}}$ und $R_{\text{g}}$
und vergleichen Sie die Effizienz.
\end{enumerate}
Zusatz: Implementieren Sie eine Kontroll-Funktion, die den aktuellen
Pfad auf die Einhaltung der Regeln des Zufallspfades �berpr�ft (sanity
check).
\end{document}
