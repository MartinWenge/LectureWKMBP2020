%% LyX 2.3.4.4 created this file.  For more info, see http://www.lyx.org/.
%% Do not edit unless you really know what you are doing.
\documentclass[12pt,german,journal=mamobx,manuscript=article,maxauthors=15,biblabel=plain]{article}
\usepackage[T1]{fontenc}
\usepackage[latin9]{inputenc}
\usepackage[a4paper]{geometry}
\geometry{verbose,tmargin=1in,bmargin=1in,lmargin=1in,rmargin=1in,headheight=0.5cm,headsep=0.5cm,footskip=0.5cm}
\setlength{\parskip}{\smallskipamount}
\setlength{\parindent}{0pt}
\usepackage{amsmath}

\makeatletter
%%%%%%%%%%%%%%%%%%%%%%%%%%%%%% User specified LaTeX commands.
\usepackage{babel}

\makeatother

\usepackage{babel}
\begin{document}

\section*{Leistungsnachweis 1: }

\subsection*{Monte-Carlo Simulationen}

Nutzen Sie die Programme der voherigen �bungen um die folgenden Aufgaben
innerhalb von 90 Minuten zu l�sen. F�r die Bewertung stellen Sie ein
\textit{kurzes} Protokoll mit dem groben Arbeitsablauf und den Antworten
zu den Fragen zusammen. Binden Sie ihre Simulationsergebnisse entsprechend
als Bilder, Diagramme oder Tabellen ein. Das Protokoll und der Quellcode
ist am Ende der Einheit per mail abzugeben.

\subsubsection*{Self Avoiding Walk}
\begin{enumerate}
\item Erschaffen Sie mit einem geeigneten Programm eine gro�e Anzahl an
selbstvermeidenden Zufallspfaden (SAW) f�r mindestens 6 verschiedene
Schrittanzahlen $N\in[16-512]$ .
\item Berechnen Sie End-zu-End Abst�nde und Gyrationsradien der SAW und
vergleichen Sie mit theoretischen Vorhersagen. Welche fraktale Dimension
haben Ihre Systeme? Welches Polymermodell zeigt die gleiche fraktale
Dimension?
\item Berechnen Sie die Streufunktion $S(q)$ um die in Aufgabe 2 getroffenen
Aussagen mit einer zweiten Methode zu verifizieren. Vergleichen Sie
die Ergebnisse f�r die fraktale Dimension. Diskutieren Sie die Vor-
und Nachteile beider Methoden
\end{enumerate}

\subsubsection*{Polymerb�rsten}
\begin{enumerate}
\item Nutzen Sie die das BFM zur Erstellung von Polymerketten mit ausgeschlossenem
Volumen, deren erstes Monomer an der (nicht periodischen) Grundfl�che
der Simulationsbox verankert ist, als Modellsystem f�r eine Polymerb�rste.
Erzeugen Sie Systeme mit $8x8$ Ketten aus 32 Monomeren bei Ankerpunktdichten
$\sigma$ 1/9, 1/25, 1/256 sowie eine einzelne geankerte Kette.
\item Equilibrieren Sie die B�rsten und bestimmen Sie die mittlere H�he
$H$. Vergleichen Sie die Abh�ngigkeit $H\sim\sigma$ mit der theoretischen
Vorhersage.
\end{enumerate}

\end{document}
